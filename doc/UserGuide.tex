\documentclass[10pt,a4paper]{article}
\usepackage[utf8]{inputenc}
\usepackage{amsmath}
\usepackage{amsfonts}
\usepackage{amssymb}
\usepackage{graphicx}
\usepackage{bm}
\usepackage[%
  colorlinks=true,
  urlcolor=blue,
  linkcolor=blue,
  citecolor=blue
]{hyperref}
\title{SsTC library\\
\large{User's Guide}\\
\large{Version 0.2.0}}
\author{Álvaro R. Puente-Uriona}
\date{\today}
\begin{document}
\maketitle
\tableofcontents
\section{Scope}
Solid-state Task Constructor (SsTC) is a hight performance computing (HPC) oriented Fortran library which aims to aid programmers and researchers in the process of sampling and integration of functions in the first Brillouin zone (BZ) of a given crystalline system. The library is centred around the following concepts.
\begin{itemize}
\item The calculator: A general function of the form
\begin{equation}\label{eq:calculator}
C^{\bm{\alpha}}(\bm{k}; \bm{\beta}),
\end{equation}
where $\bm{k}$ is a vector in the BZ and $\bm{\alpha}, \bm{\beta}$ encompass \textit{all} other functional dependences of $C$. In the document we refer to $\bm{\alpha}$ as integer or discrete indices and seek to represent, for example, a set of Cartesian components $\{x, y, z\}$ or band indices. On the other side, $\bm{\beta}$ are referred as continuous indices and seek to represent the dependence of $C$ on non-intrinsic variables such as an externally controlled frequency $\omega$ or a variable range of Fermi energies $\varepsilon_F$.
\item The task: An object containing a complete description of the calculator $C$ and the sampling or integration task that the programmer wants to make.
\item The system: An object containing a complete description of a crystalline system given by its tight-binding \cite{marzariMaximallyLocalizedWannier2012} representation. This includes:
\begin{itemize}
\item Number of bands.
\item Fermi energy.
\item Unit cell description: the projection of vectors $\bm{a}_i$ in the $\{x, y, z\}$ axes in units of \r{A}.
\item Resolution of the Hamiltonian operator $\hat{H}$ in the basis of the Wannier functions (WFs) \cite{marzariMaximallyLocalizedWannier2012} in units of eV, i.e., the on-site and tunnelling amplitudes.
\item Resolution of the position operator $\hat{\bm{r}}$ in the basis of the WFs in units of \r{A}.
\end{itemize}
\end{itemize}
The sampling and integration utilities contained in SsTC are the following.
\begin{itemize}
\item ``Kpath" sampling. 1-dimensional sampling of $C$ with specification of the path to sample.
\item ``Kslice" sampling. 2-dimensional regular sampling of $C$ with specification of the slice to sample.
\item ``Regular" sampling. 3-dimensional regular sampling of $C$ with specification of the number of samples in each dimension.
\item Integration. 3-dimensional regular sampling of $C$ followed by an integration process. The integration process as for v0.2.0 amounts to the rectangle approximation in each dimension. There exists also an experimental integration method based on extrapolation methods.
\end{itemize}
All the concepts covered in this section are treated with greater detail in the specific sections in the document.
\section{Installation \& linking to application}
As for v0.2.0, the library is only guaranteed to compile on Linux systems using the \verb|mpiifort| and \verb|mpiifx| compilers contained in the Intel\textregistered oneAPI HPC toolkit, \href{https://www.intel.com/content/www/us/en/developer/tools/oneapi/hpc-toolkit.html}{external link}. As such, the main application to be linked with must be compiled with \verb|mpiifort| or \verb|mpiifx|.

The SsTC project is hosted on \href{https://github.com/irukoa/SsTC}{GitHub} and can be downloaded by running the command
\begin{verbatim}
git clone --recurse-submodules https://github.com/irukoa/SsTC.git
\end{verbatim}
or by downloading a compressed version of the source code on the \href{https://github.com/irukoa/SsTC/tags}{tag history}.

To install the source code move the \verb|SsTC| folder to a path of your choice and run \verb|make|.
\begin{verbatim}
bash:/current/dir$ mv SsTC/ /path/of/your/choice/
bash:/path/of/your/choice$ cd SsTC/
bash:/path/of/your/choice/SsTC$ make
\end{verbatim}
These commands will compile the source code and install the library \verb|libSsTC.a| and the module header file \verb|sstc.mod| in \verb|/path/of/your/choice/SsTC/bin/|.

To link to your Fortran application \verb|appl.F90|, add the line \verb|use SsTC| in your application preamble and compile with 
\begin{verbatim}
bash:/path/to/application/$ $(F90) $(F90FLAGS) appl.F90 
-I/path/of/your/choice/SsTC/bin 
/path/of/your/choice/SsTC/bin/libSsTC.a -o "appl.x"
\end{verbatim}
where \verb|F90 = mpiifort/mpiifx| and \verb|F90FLAGS| contains, at least,
\begin{verbatim}
F90FLAGS = -qopenmp -lmkl_intel_lp64 -lmkl_core -lmkl_gnu_thread 
-pthread
\end{verbatim}

To make use of SsTC, the application preamble should contain also the lines 
\begin{verbatim}
use MPI_F08
use OMP_LIB
\end{verbatim}
The application must contain also a call to the MPI \cite{messagepassinginterfaceforumMPIMessagePassingInterface2021} initialization routine \verb|call MPI_INIT(ierror)| and to the SsTC initialization routine \\ \verb|call SsTC_init()| \textit{before} any call to other SsTC routines.

Note:  By default SsTC uses double precision \verb|dp, kind=8| numbers except for integers, which are the default \verb|kind=4|.
\section{Calculators \& tasks}
The calculator Eq. \eqref{eq:calculator} is the generic name for Fortran function with interface
\begin{verbatim}
abstract interface
  function SsTC_global_calculator(task, system, k, error) &
           result(u)
    class(SsTC_global_k_data), intent(in) :: task
    type(SsTC_sys), intent(in)            :: system
    real(kind=dp), intent(in)             :: k(3)
    logical, intent(inout)                :: error

    complex(kind=dp) :: u(product(task%integer_indices), &
                          product(task%continuous_indices))
  end function SsTC_global_calculator
end interface
\end{verbatim}
given by one of the SsTC modules, see Sec. \ref{sec:modularity}, or otherwise provided by the user in the scope of the main application using SsTC. The calculator must be interface conforming.

The task is a Fortran object (derived type) specifying the \\ sampling/integration task to perform and contains a full description of the properties of the calculator. The generic properties which applies to every task \verb|class(SsTC_global_k_data) :: task| are the following,
\begin{itemize}
\item \verb|character(len=*) :: task%name|: The name given to the task.
\item \verb|procedure(SsTC_global_calculator), &|\\
	  \verb|pointer :: task%global_calculator|: Interface conforming procedure pointer to the selected calculator. The library's sampling routines will use the provided calculator to sample.
\item \verb|integer :: task%integer_indices(N_int_ind)|: Each entry \verb|i| of the array contains the number of values the integer index $\alpha_i$ in Eq. \eqref{eq:calculator} can have. \verb|N_int_ind| is the total number of integer indices encompassed by $\bm{\alpha}$ and the total number of integer index combinations is given by \verb|product(task%integer_indices)|.
\item \verb|integer :: task%continuous_indices(N_ext_vars)|: Each entry \verb|i| of the array contains the number \verb|ext_vars_steps(i)| of values the continuous index $\beta_i$ in Eq. \eqref{eq:calculator} can have. \verb|N_ext_vars| is the total number of external variables encompassed by $\bm{\beta}$ and the total number of continuous index combinations is given by \verb|product(task%continuous_indices)|.
\item \verb|real(kind=dp) :: task%ext_var_data(N_ext_vars)%&|\\
      \verb|data(ext_vars_steps(N_ext_vars))|: Each entry \verb|i, j| corresponding to \verb|task%ext_var_data(i)%data(j)| is a real number $\lambda_{ij}$ containing the particular value $\beta_{i}$ has, as given by
\begin{equation}
\lambda_{ij} = \lambda_{i1} + \left(\lambda_{i\;\verb|ext_vars_steps(i)|} - \lambda_{i1}\right)\times\left(j - 1\right)/\verb|ext_vars_steps(i)|
\end{equation}
where $\lambda_{i1}$ and $\lambda_{i\;\verb|ext_vars_steps(i)|}$ are the starting and ending points of the values given to $\beta_{i}$.
\end{itemize}
\section{Systems}
\section{Sampling and integration routines}
\subsection{Kpath module}
\subsection{Kslice module}
\subsection{Sampler module}
\subsection{Integrator module}
\section{Usage in high performance computing}
\section{Other routines}
\subsection{Comms module}
\subsection{Utility module}
\subsection{Extrapolation integration module}
\subsection{Data structures module}
\subsection{Local k-quantities module}
\section{Modularity}\label{sec:modularity}
\section{Examples}
\nocite{*}
\bibliographystyle{unsrt}
\bibliography{bibdata}
\end{document}
